% Uncomment for handout
%\def\HANDOUT{}


\ifdefined\HANDOUT
\documentclass[handout]{beamer}
\usepackage{pgfpages}
\pgfpagesuselayout{4 on 1}[letterpaper,landscape,border shrink=5mm]
\else
\documentclass{beamer}
\fi

\mode<presentation>
{
  \usetheme{Warsaw}
  \definecolor{sered}{rgb}{0.78, 0.06, 0.18}
  \definecolor{richblack}{rgb}{0.0, 0.0, 0.0}
  \setbeamercolor{structure}{fg=sered,bg=richblack}
  %\setbeamercovered{transparent}
}


\usepackage[english]{babel}
\usepackage[latin1]{inputenc}
\usepackage{times}
\usepackage[T1]{fontenc}
\usepackage{tikz}
\usepackage{graphicx}
\usepackage[export]{adjustbox}
\usepackage{fancyvrb}
\usepackage{amsmath}
\usepackage{amssymb}
\usepackage{esvect}

\makeatletter
\newcommand{\imagesource}[1]{{\centering\hfill\break\hbox{\scriptsize Image Source:\thinspace{\tiny\itshape #1}}\par}}
\newcommand{\image}[3][\@nil]{%
        \def\tmp{#1}%
        \begin{center}
        \ifx\tmp\@nnil
            \includegraphics[max height = 0.55\textheight, max width = \textwidth]{images/#2}
        \else
            \includegraphics[max height = 0.50\textheight, max width = \textwidth]{images/#2}
            \linebreak
            #1
        \fi
        \linebreak
        {\tiny Image Source:\thinspace{\tiny #3}}
        \end{center}
}

\newenvironment{code}{%
 \VerbatimEnvironment
 \begin{adjustbox}{max width=\textwidth, max height=0.7\textheight}
 \begin{BVerbatim}
  }{
  \end{BVerbatim}
 \end{adjustbox}
}

\title{Research Schedule and Design}


\author{Robert Lowe}

\institute[Southeast Missouri State University] % (optional, but mostly needed)
{
  Department of Computer Science\\
  Southeast Missouri State University
}

\date[]{}
\subject{}

\pgfdeclareimage[height=1.0cm]{university-logo}{images/semo-logo}
\logo{\pgfuseimage{university-logo}}



\AtBeginSection[]
{
  \begin{frame}<beamer>{Outline}
    \tableofcontents[currentsection]
  \end{frame}
}


\begin{document}

\begin{frame}
  \titlepage
\end{frame}

\begin{frame}{Outline}
  \tableofcontents
\end{frame}


% Structuring a talk is a difficult task and the following structure
% may not be suitable. Here are some rules that apply for this
% solution: 

% - Exactly two or three sections (other than the summary).
% - At *most* three subsections per section.
% - Talk about 30s to 2min per frame. So there should be between about
%   15 and 30 frames, all told.

% - A conference audience is likely to know very little of what you
%   are going to talk about. So *simplify*!
% - In a 20min talk, getting the main ideas across is hard
%   enough. Leave out details, even if it means being less precise than
%   you think necessary.
% - If you omit details that are vital to the proof/implementation,
%   just say so once. Everybody will be happy with that.

\section{Schedule}
\begin{frame}{Typical Schedule Components}
\begin{columns}
    \column{0.3\textwidth}
    \begin{itemize}
        \item Timeline
        \item Milestones
        \item Budget
    \end{itemize}
    
    
    \column{0.7\textwidth}
    \image[Gantt Chart]{gantt-chart}{\href{https://ccts.osu.edu/sites/default/files/inline-files/Gantt\%20Chart\%20Example.pdf}{Project Management for Research}}
\end{columns}
\end{frame}

\begin{frame}{Timeline}
    \begin{itemize}
        \item When do we expect to have deliverables?
        \item When will we publish?
        \item How many papers do we expect to produce?
    \end{itemize}
\end{frame}

\begin{frame}{Milestones}
    \begin{itemize}
        \item What are the major elements of our research?
        \item What accomplishments will we make over time?
        \item How long do we expect each phase to last?
        \item In what order must the milestones be accomplished?
        \item How are the milestones interrelated?
    \end{itemize}
\end{frame}

\begin{frame}{Budget}
    \begin{itemize}
        \item How much will the research cost?
        \item How will the researchers be paid?
        \item What are the material costs of this research?
    \end{itemize}
\end{frame}

\section{Design}

\begin{frame}{Design Goals}
    A good research project is:
    \begin{itemize}
        \item Verifiable
        \item Repeatable
        \item Statistically Sound
    \end{itemize}
\end{frame}

\begin{frame}{Design Questions}
    \begin{itemize}
        \item How will we gather data?
        \item What sort of data will we gather?
        \item How will we evaluate the data?
    \end{itemize}
\end{frame}

\begin{frame}{Data and Code Publication}
    \begin{itemize}
        \item Increasingly, researchers are expected to make their code public.
        \item Several code and data venues exist.
    \end{itemize}
\end{frame}

\end{document}
