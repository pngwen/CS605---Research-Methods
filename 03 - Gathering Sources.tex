% Uncomment for handout
%\def\HANDOUT{}


\ifdefined\HANDOUT
\documentclass[handout]{beamer}
\usepackage{pgfpages}
\pgfpagesuselayout{4 on 1}[letterpaper,landscape,border shrink=5mm]
\else
\documentclass{beamer}
\fi

\mode<presentation>
{
  \usetheme{Warsaw}
  \definecolor{sered}{rgb}{0.78, 0.06, 0.18}
  \definecolor{richblack}{rgb}{0.0, 0.0, 0.0}
  \setbeamercolor{structure}{fg=sered,bg=richblack}
  %\setbeamercovered{transparent}
}


\usepackage[english]{babel}
\usepackage[latin1]{inputenc}
\usepackage{times}
\usepackage[T1]{fontenc}
\usepackage{tikz}
\usepackage{graphicx}
\usepackage[export]{adjustbox}
\usepackage{fancyvrb}
\usepackage{amsmath}
\usepackage{amssymb}
\usepackage{esvect}

\makeatletter
\newcommand{\imagesource}[1]{{\centering\hfill\break\hbox{\scriptsize Image Source:\thinspace{\tiny\itshape #1}}\par}}
\newcommand{\image}[3][\@nil]{%
        \def\tmp{#1}%
        \begin{center}
        \ifx\tmp\@nnil
            \includegraphics[max height = 0.55\textheight, max width = \textwidth]{images/#2}
        \else
            \includegraphics[max height = 0.50\textheight, max width = \textwidth]{images/#2}
            \linebreak
            #1
        \fi
        \linebreak
        {\tiny Image Source:\thinspace{\tiny #3}}
        \end{center}
}

\newenvironment{code}{%
 \VerbatimEnvironment
 \begin{adjustbox}{max width=\textwidth, max height=0.7\textheight}
 \begin{BVerbatim}
  }{
  \end{BVerbatim}
 \end{adjustbox}
}

\title{03 - Gathering Sources}


\author{Robert Lowe}

\institute[Southeast Missouri State University] % (optional, but mostly needed)
{
  Department of Computer Science\\
  Southeast Missouri State University
}

\date[]{}
\subject{}

\pgfdeclareimage[height=1.0cm]{university-logo}{images/semo-logo}
\logo{\pgfuseimage{university-logo}}



\AtBeginSection[]
{
  \begin{frame}<beamer>{Outline}
    \tableofcontents[currentsection]
  \end{frame}
}


\begin{document}

\begin{frame}
  \titlepage
\end{frame}


% Structuring a talk is a difficult task and the following structure
% may not be suitable. Here are some rules that apply for this
% solution: 

% - Exactly two or three sections (other than the summary).
% - At *most* three subsections per section.
% - Talk about 30s to 2min per frame. So there should be between about
%   15 and 30 frames, all told.

% - A conference audience is likely to know very little of what you
%   are going to talk about. So *simplify*!
% - In a 20min talk, getting the main ideas across is hard
%   enough. Leave out details, even if it means being less precise than
%   you think necessary.
% - If you omit details that are vital to the proof/implementation,
%   just say so once. Everybody will be happy with that.

\begin{frame}{What is the purpose of research?}
    \begin{columns}
    \column{0.6\textwidth}
        \begin{block}{}
        If I have seen further it is by standing on the shoulders of giants.
        \linebreak
        {\scriptsize--Sir Isaac Newton, 1676 Letter to Robert Hooke}
        \end{block}
        
        \begin{itemize}
            \item The purpose of research is to increase the sum of human knowledge.
            \item Each paper builds on past papers.
            \item Each paper must be unique, it must contribute something new.
        \end{itemize}
        
    \column{0.4\textwidth}
    \image[Sir Isaac Newton\linebreak (1643 - 1727)]{newton}{\href{https://en.wikipedia.org/wiki/Isaac_Newton}{Wikipedia}}
    \end{columns}
\end{frame}

\begin{frame}{Finding Sources}
    \begin{itemize}
        \item Graduate students once practically lived in libraries.
        \item Today, research is done online through various electronic databases.
        \item Two important sources for you:
        \begin{itemize}
           \item \href{https://scholar.google.com}{scholar.google.com}
           \item \href{https://library.semo.edu/find-materials/articles-and-databases}{Kent Library Articles and Databases}
        \end{itemize}
    \end{itemize}
\end{frame}

\begin{frame}{Some Prominant CS Databases}
    \begin{description}
        \item[ACM] - Association for Computing Machinery
        \item[IEEE] - The Institute of Electrical and Electronics Engineers 
        \item[SIAM] - Society for Industrial and Applied Mathematics
        \item[Springer] - Publisher of Many Scientific and Technical Works
    \end{description}
\end{frame}

\begin{frame}{Tools For Collaboration}
    To help you work with your groups, I have created two primary means of collaboration:
    \begin{itemize}
        \item Microsoft Teams Group Channel
        \begin{itemize}
            \item Share Files
            \item Text Chat
            \item Video Chat
        \end{itemize}
        \item Overleaf Group Project
        \begin{itemize}
            \item Write your LateX documents.
            \item Chat
            \item I can see your writing to help with \LaTeX questions.
        \end{itemize}
    \end{itemize}
\end{frame}

\begin{frame}{Assignment}
    \begin{itemize}
        \item \href{https://semo.instructure.com/courses/9573/assignments/82356?module_item_id=340086}{Gathering Sources}
    \end{itemize}
\end{frame}


\end{document}
