
% Uncomment for handout
%\def\HANDOUT{}


\ifdefined\HANDOUT
\documentclass[handout]{beamer}
\usepackage{pgfpages}
\pgfpagesuselayout{4 on 1}[letterpaper,landscape,border shrink=5mm]
\else
\documentclass{beamer}
\fi

\mode<presentation>
{
  \usetheme{Warsaw}
  \definecolor{sered}{rgb}{0.78, 0.06, 0.18}
  \definecolor{richblack}{rgb}{0.0, 0.0, 0.0}
  \setbeamercolor{structure}{fg=sered,bg=richblack}
  %\setbeamercovered{transparent}
}


\usepackage[english]{babel}
\usepackage[latin1]{inputenc}
\usepackage{times}
\usepackage[T1]{fontenc}
\usepackage{tikz}
\usepackage{graphicx}
\usepackage[export]{adjustbox}
\usepackage{fancyvrb}
\usepackage{amsmath}
\usepackage{amssymb}

\newcommand{\imagesource}[1]{{\centering\hfill\break\hbox{\scriptsize Image Source:\thinspace{\tiny\itshape #1}}\par}}
\newcommand{\image}[2]{%
        \begin{center}
        \includegraphics[max height = 0.55\textheight, max width = \textwidth]{images/#1}
        \linebreak
        {\tiny Image Source:\thinspace{\tiny #2}}
        \end{center}
}

\newenvironment{code}{%
 \VerbatimEnvironment
 \begin{adjustbox}{max width=\textwidth, max height=0.7\textheight}
 \begin{BVerbatim}
  }{
  \end{BVerbatim}
 \end{adjustbox}
}

\title{02 - Working With \LaTeX}


\author{Robert Lowe}

\institute[Southeast Missouri State University] % (optional, but mostly needed)
{
  Department of Computer Science\\
  Southeast Missouri State University
}

\date[]{}
\subject{}

\pgfdeclareimage[height=1.0cm]{university-logo}{images/semo-logo}
\logo{\pgfuseimage{university-logo}}



\AtBeginSection[]
{
  \begin{frame}<beamer>{Outline}
    \tableofcontents[currentsection]
  \end{frame}
}


\begin{document}

\begin{frame}
  \titlepage
\end{frame}

\begin{frame}{Outline}
  \tableofcontents
\end{frame}


% Structuring a talk is a difficult task and the following structure
% may not be suitable. Here are some rules that apply for this
% solution: 

% - Exactly two or three sections (other than the summary).
% - At *most* three subsections per section.
% - Talk about 30s to 2min per frame. So there should be between about
%   15 and 30 frames, all told.

% - A conference audience is likely to know very little of what you
%   are going to talk about. So *simplify*!
% - In a 20min talk, getting the main ideas across is hard
%   enough. Leave out details, even if it means being less precise than
%   you think necessary.
% - If you omit details that are vital to the proof/implementation,
%   just say so once. Everybody will be happy with that.

\section{History}

\begin{frame}{The Art of Computer Programming}
    \begin{columns}
    \column{0.6\textwidth}
    \begin{itemize}
        \item Written by Donald Knuth, first published in 1968.
        \item First edition was typeset using traditional hot metal techniques.
        \item The Second Edition (1976) was typeset using Photo Typesetting, which yielded inferior results.
        \item Knuth set out to write a digital typesetting system to improve modern typesetting.
    \end{itemize}
    
    \column{0.4\textwidth}
        \image{aocp}{\href{https://www.amazon.com/Computer-Programming-Volumes-1-4A-Boxed/dp/0321751043/ref=sr_1_2?dchild=1&keywords=the+art+of+computer+programming&qid=1624915066&sr=8-2}{amazon.com}}
    \end{columns}
\end{frame}

\begin{frame}{TeX}
    \begin{columns}
    \column{0.6\textwidth}
        \begin{itemize}
            \item Macro Language
            \item Goals
            \begin{itemize}
                \item Allow anyone to produce high-quality books.
                \item Have documents give the same result when compiled on all computers.
            \end{itemize}
            \item Developed by Donald Knuth, first released in 1978.
            \item Official Website: \url{http://tug.org/}
        \end{itemize}
    \column{0.4\textwidth}
        \image{knuth}{\href{https://en.wikipedia.org/wiki/Donald_Knuth\#/media/File:KnuthAtOpenContentAlliance.jpg}{Wikipedia}}
    \end{columns}
\end{frame}


\begin{frame}{\LaTeX}
    \begin{columns}
    \column{0.6\textwidth}
    \begin{itemize}
        \item A set of macros written in TeX by Leslie Lamport in the early 1980s.
        \item Provides Structure to TeX Documents
        \item Popular for academic, especially scientific, writing.
    \end{itemize}
    \column{0.4\textwidth}
        \image{lamport}{\href{https://en.wikipedia.org/wiki/Leslie_Lamport\#/media/File:Leslie_Lamport.jpg}{Wikipedia}}
    \end{columns}
\end{frame}


\begin{frame}{\LaTeX in Modern Usage}
    \begin{columns}
    \column{0.6\textwidth}
    \begin{itemize}
        \item Many journals and book publishers require \LaTeX!
        \item Almost all will accept \LaTeX.
        \item Most of your textbooks were likely typeset in \LaTeX!
    \end{itemize}
    \column{0.4\textwidth}
        \includegraphics[max width=\textwidth]{images/latex-usage}
    \end{columns}
\end{frame}

\section{\LaTeX in Action}

\begin{frame}{Why should I learn \LaTeX?}
    \begin{enumerate}
        \item Its output is gorgeous!
        \item Focus on structure, not formatting.
        \item Precise formatting.
        \item Easily change document's style to fit publishing requirements.
        \item Manage figures, tables, table of contents, etc. with ease!
        \item Manage bibliographies easily through bibtex.
        \item Once learned, it is faster than WYSWIG solutions.
        \item And more...
    \end{enumerate}
\end{frame}

\begin{frame}[fragile]{Document Structure}
    \begin{columns}
    \column{0.5\textwidth}
    Every \LaTeX~document contains two parts:
    \begin{enumerate}
        \item Preamble
        \item Document Body
    \end{enumerate}
    
    \column{0.5\textwidth}
    \begin{code}
% Preamble
\documentclass{article}
\title{ Really Clever Research }
\author{ Dr. Scientist }

\begin{document}
% Document body
\maketitle
\tableofcontents

\section{Introduction}
Blah blah blah blah...

Blah blah blah blah...

\section{Conclusions}
En vino veritas!

Ad curram et communico.


\end{document}
    \end{code}
    \end{columns}
\end{frame}


\begin{frame}[fragile]{Environments}
    Environments use begin and end commands, and provide powerful formatting.
    \begin{columns}
    \column{0.5\textwidth}
    \begin{code}
    \begin{align*}
        x^2+4 &= 20\\
        x^2+4 - 4 &= 20 - 4\\
        x^2 &= 16\\
        \sqrt{x^2} &= \sqrt{16} \\
        x &= 4\\
    \end{align*}
    
    \end{code}
    \column{0.5\textwidth}
    \begin{align*}
        x^2+4 &= 20\\
        x^2+4 - 4 &= 20 - 4\\
        x^2 &= 16\\
        \sqrt{x^2} &= \sqrt{16} \\
        x &= 4\\
    \end{align*}
    \end{columns}
\end{frame}

\begin{frame}[fragile]{Typesetting Math}
    Formula typesetting is where TeX and \LaTeX really shine!
    \begin{columns}
    \column{0.5\textwidth}
    \begin{code}
        \[
        f(x) = \left\{
        \begin{array}{ll}
            1 & x \leq 2 \\
            f(x-1) + f(x-2) & \mathrm{otherwise}
        \end{array}
        \right.
        \]
    \end{code}
    \column{0.5\textwidth}
        \begin{adjustbox}{max width=\textwidth}
        \[
        f(x) = \left\{
        \begin{array}{ll}
            1 & x \leq 2 \\
            f(x-1) + f(x-2) & \mathrm{otherwise}
        \end{array}
        \right.
        \]
        \end{adjustbox}
    \end{columns}
\end{frame}


\begin{frame}[fragile]{Managing Citations}
    \begin{columns}
    \column{0.5\textwidth}
        \begin{itemize}
            \item Internal References via ref command.
            \item Bibliography citations via cite command.
            \item Citation databases through BibTeX.
            \item BibTex citations are provided by many scholarly databases.
            \item Let's look at \href{https://scholar.google.com}{scholar.google.com}
        \end{itemize}
    \column{0.5\textwidth}
    \begin{code}
@inproceedings{johnson2009client,
  title={A client-server system for simulating and 
         visualizing the  random packing of polydiverse spheres},
  author={Johnson, Bruce and Lowe, Robert},
  booktitle={IEEE Southeastcon 2009},
  pages={307--310},
  year={2009},
  organization={IEEE}
}
    \end{code}
    \end{columns}    
\end{frame}


\begin{frame}{Assignment}
    \begin{itemize}
        \item\href{https://canvas.semo.edu/courses/9573/assignments/82228?module_item_id=339519}{Working with Latex}
    \end{itemize}
\end{frame}

\end{document}
