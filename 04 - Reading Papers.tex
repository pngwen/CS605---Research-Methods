% Uncomment for handout
\def\HANDOUT{}


\ifdefined\HANDOUT
\documentclass[handout]{beamer}
\usepackage{pgfpages}
\pgfpagesuselayout{4 on 1}[letterpaper,landscape,border shrink=5mm]
\else
\documentclass{beamer}
\fi

\mode<presentation>
{
  \usetheme{Warsaw}
  \definecolor{sered}{rgb}{0.78, 0.06, 0.18}
  \definecolor{richblack}{rgb}{0.0, 0.0, 0.0}
  \setbeamercolor{structure}{fg=sered,bg=richblack}
  %\setbeamercovered{transparent}
}


\usepackage[english]{babel}
\usepackage[latin1]{inputenc}
\usepackage{times}
\usepackage[T1]{fontenc}
\usepackage{tikz}
\usepackage{graphicx}
\usepackage[export]{adjustbox}
\usepackage{fancyvrb}
\usepackage{amsmath}
\usepackage{amssymb}

\newcommand{\imagesource}[1]{{\centering\hfill\break\hbox{\scriptsize Image Source:\thinspace{\tiny\itshape #1}}\par}}
\newcommand{\image}[2]{%
        \begin{center}
        \includegraphics[max height = 0.55\textheight, max width = \textwidth]{images/#1}
        \linebreak
        {\tiny Image Source:\thinspace{\tiny #2}}
        \end{center}
}

\newenvironment{code}{%
 \VerbatimEnvironment
 \begin{adjustbox}{max width=\textwidth, max height=0.7\textheight}
 \begin{BVerbatim}
  }{
  \end{BVerbatim}
 \end{adjustbox}
}

\title{04 - Reading Papers}


\author{Robert Lowe}

\institute[Southeast Missouri State University] % (optional, but mostly needed)
{
  Department of Computer Science\\
  Southeast Missouri State University
}

\date[]{}
\subject{}

\pgfdeclareimage[height=1.0cm]{university-logo}{images/semo-logo}
\logo{\pgfuseimage{university-logo}}



\AtBeginSection[]
{
  \begin{frame}<beamer>{Outline}
    \tableofcontents[currentsection]
  \end{frame}
}


\begin{document}

\begin{frame}
  \titlepage
\end{frame}

% Structuring a talk is a difficult task and the following structure
% may not be suitable. Here are some rules that apply for this
% solution: 

% - Exactly two or three sections (other than the summary).
% - At *most* three subsections per section.
% - Talk about 30s to 2min per frame. So there should be between about
%   15 and 30 frames, all told.

% - A conference audience is likely to know very little of what you
%   are going to talk about. So *simplify*!
% - In a 20min talk, getting the main ideas across is hard
%   enough. Leave out details, even if it means being less precise than
%   you think necessary.
% - If you omit details that are vital to the proof/implementation,
%   just say so once. Everybody will be happy with that.
\section{Reading Papers}
\begin{frame}{What to Avoid}
    \image{conspiracy}{It's Always Sunny in Philadelphia, Season 4 Episode 10}
\end{frame}

\begin{frame}{Typical Layout of a Scholarly Paper}
    \begin{enumerate}
        \item Abstract
        \item Introduction 
        \item Related Works
        \item Method / Experiments
        \item Conclusions
        \item Bibliography
    \end{enumerate}
\end{frame}

\begin{frame}{Reading Sequence}
    \begin{block}{}
        {\bf Pro Tip:} Scholarly articles are functional documents. 
        They are not typically meant to be consumed like magazines or books!
    \end{block}
    
    \begin{enumerate}
        \item {\bf Quick Scan} - Determine what this paper says, and determine if it is
          relevant.
        \item {\bf Citation Exploration} - Determine how this paper fits into its body
          of literature. Identify additional papers to read.
        \item {\bf In-Depth Reading} - Study the paper, especially the methodology. This 
          is where you learn the details of the author's contribution.
    \end{enumerate}
\end{frame}

\begin{frame}{Quick Scan}

    {\bf Procedure}
    \begin{enumerate}
        \item Read the abstract.
        \item Read the conclusions.
    \end{enumerate}
    
    \vspace{1em}
    
    {\bf Goals}
    \begin{itemize}
        \item Determine relevance.
        \item Determine importance of the results.
        \item Identify the author's claimed contributions.
    \end{itemize}
    
\end{frame}


\begin{frame}{Citation Exploration}

    {\bf Procedure}
    \begin{enumerate}
        \item Read the Introduction.
        \item Read the Related Works.
        \item Scan the Bibliography.
    \end{enumerate}
    
    \vspace{1em}
    
    {\bf Goals}
    \begin{itemize}
        \item Determine how this paper's conclusions follow from previous work.
        \item Identify other papers that may be relevant to your own work.
        \item Identify which papers are likely to be needed to provide details for
          the current paper's methodology.
    \end{itemize}
\end{frame}

\begin{frame}{In-Depth Reading}

    {\bf Procedure}
    \begin{enumerate}
        \item Read the Method / Experiment Section.
        \item Re-Read the Conclusion.
        \item Read any papers needed to clarify the methodology.
    \end{enumerate}
    
    \vspace{1em}
    
    {\bf Goals}
    \begin{itemize}
        \item Gain an understanding of what methods the author used.
        \item Determine how well the experiment supports the conclusions.
        \item Try to form ideas about the following:
        \begin{itemize}
            \item Is this method directly applicable to my current work?
            \item Can I somehow improve upon the author's method?
        \end{itemize}
    \end{itemize}
    
\end{frame}

\section{Taking Notes}
\begin{frame}{Taking Notes}
    \begin{block}{}
        Throughout all your reading, you should be taking and gathering notes.
    \end{block}
    
    \begin{enumerate}
        \item Annotate the papers themselves.
        \item Gather notes into one location.
    \end{enumerate}
\end{frame}

\begin{frame}{Annotation}
    \begin{itemize}
        \item Use a tool such as OneNote to directly annotate papers.
        \item Highlight sections relevant to you.
        \item Add your own handwritten notes in margins.
    \end{itemize}
\end{frame}

\begin{frame}{Note Gathering}
    \begin{itemize}
        \item Annotations on papers are not sufficient.
        \item Gather your notes into one document, typically typeset in latex.
        \item Your typeset notes should contain:
        \begin{itemize}
            \item Summaries and paraphrases of your annotations.
            \item Typeset any relevant formulae from the paper.
            \item A few sentences summarizing the paper.
            \item A few sentences summarizing the relevance of this paper to your own work.
            \item A list of references from this paper which you plan to read.
            \item Any other relevant information you have gained from this paper.
        \end{itemize}
    \end{itemize}
\end{frame}

\end{document}
