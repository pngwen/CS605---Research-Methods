% Uncomment for handout
%\def\HANDOUT{}


\ifdefined\HANDOUT
\documentclass[handout]{beamer}
\usepackage{pgfpages}
\pgfpagesuselayout{4 on 1}[letterpaper,landscape,border shrink=5mm]
\else
\documentclass{beamer}
\fi

\mode<presentation>
{
  \usetheme{Warsaw}
  \definecolor{sered}{rgb}{0.78, 0.06, 0.18}
  \definecolor{richblack}{rgb}{0.0, 0.0, 0.0}
  \setbeamercolor{structure}{fg=sered,bg=richblack}
  %\setbeamercovered{transparent}
}


\usepackage[english]{babel}
\usepackage[latin1]{inputenc}
\usepackage{times}
\usepackage[T1]{fontenc}
\usepackage{tikz}
\usepackage{graphicx}
\usepackage[export]{adjustbox}
\usepackage{fancyvrb}
\usepackage{amsmath}
\usepackage{amssymb}

\newcommand{\imagesource}[1]{{\centering\hfill\break\hbox{\scriptsize Image Source:\thinspace{\tiny\itshape #1}}\par}}
\newcommand{\image}[2]{%
        \begin{center}
        \includegraphics[max height = 0.55\textheight, max width = \textwidth]{images/#1}
        \linebreak
        {\tiny Image Source:\thinspace{\tiny #2}}
        \end{center}
}

\newenvironment{code}{%
 \VerbatimEnvironment
 \begin{adjustbox}{max width=\textwidth, max height=0.7\textheight}
 \begin{BVerbatim}
  }{
  \end{BVerbatim}
 \end{adjustbox}
}

\title{01 - Introduction to Research Methods}


\author{Dr. Robert Lowe}

\institute[Southeast Missouri State University] % (optional, but mostly needed)
{
  Department of Computer Science\\
  Southeast Missouri State University
}

\date[]{}
\subject{}

\pgfdeclareimage[height=1.0cm]{university-logo}{images/semo-logo}
\logo{\pgfuseimage{university-logo}}



\AtBeginSection[]
{
  \begin{frame}<beamer>{Outline}
    \tableofcontents[currentsection]
  \end{frame}
}


\begin{document}

\begin{frame}
  \titlepage
\end{frame}


% Structuring a talk is a difficult task and the following structure
% may not be suitable. Here are some rules that apply for this
% solution: 

% - Exactly two or three sections (other than the summary).
% - At *most* three subsections per section.
% - Talk about 30s to 2min per frame. So there should be between about
%   15 and 30 frames, all told.

% - A conference audience is likely to know very little of what you
%   are going to talk about. So *simplify*!
% - In a 20min talk, getting the main ideas across is hard
%   enough. Leave out details, even if it means being less precise than
%   you think necessary.
% - If you omit details that are vital to the proof/implementation,
%   just say so once. Everybody will be happy with that.

\begin{frame}{Course Overview}
    \begin{columns}
    \column{0.5\textwidth}
    \begin{itemize}
        \item Means of Communication
        \begin{itemize}
            \item Canvas
            \item Microsoft Teams
            \item Overleaf
        \end{itemize}
        \item The Course Syllabus
    \end{itemize}
    
    \column{0.5\textwidth}
    \image{research.jpg}{\href{https://www.formpl.us/blog/research-report}{formpl.us}}
    \end{columns}
\end{frame}

\begin{frame}{Research Process Overview}
    Typical Phases:
    \begin{enumerate}
        \item Literature Review
        \item Scheduling and Proposal
        \item Experimentation
        \item Writing
        \item Presentation
    \end{enumerate}
\end{frame}

\begin{frame}{Researcher Tools}
    \begin{itemize}
        \item Online Databases
        \begin{itemize}
            \item \url{https://scholar.google.com}
            \item \url{https://library.semo.edu/find-materials/articles-and-databases}
        \end{itemize}
        \item Typesetting and Writing Tools
        \begin{itemize}
            \item \LaTeX
            \item BibTeX
            \item \url{https://www.overleaf.com}
        \end{itemize}
        \item Data Processing and Visualization Tools
        \begin{itemize}
            \item Spreadsheets (Excel, Google Sheets, etc.)
            \item R and R Studio
            \item Matlab
        \end{itemize}
    \end{itemize}
\end{frame}

\begin{frame}{Effective Online Collaboration}
    \begin{itemize}
        \item Modern researchers work with colleagues from all over the world!
        \item Working remotely online has become a crucial skill.
        \item As we are an online course, we will be working exclusively online, so you will hone this skill.
    \end{itemize}
\end{frame}

\begin{frame}{Assignments}
    \begin{enumerate}
        \item Learn \LaTeX in 30 Minutes 
        \item Join a Group
    \end{enumerate}
\end{frame}


\end{document}
