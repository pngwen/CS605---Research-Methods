% Uncomment for handout
%\def\HANDOUT{}


\ifdefined\HANDOUT
\documentclass[handout]{beamer}
\usepackage{pgfpages}
\pgfpagesuselayout{4 on 1}[letterpaper,landscape,border shrink=5mm]
\else
\documentclass{beamer}
\fi

\mode<presentation>
{
  \usetheme{Warsaw}
  \definecolor{sered}{rgb}{0.78, 0.06, 0.18}
  \definecolor{richblack}{rgb}{0.0, 0.0, 0.0}
  \setbeamercolor{structure}{fg=sered,bg=richblack}
  %\setbeamercovered{transparent}
}


\usepackage[english]{babel}
\usepackage[latin1]{inputenc}
\usepackage{times}
\usepackage[T1]{fontenc}
\usepackage{tikz}
\usepackage{graphicx}
\usepackage[export]{adjustbox}
\usepackage{fancyvrb}
\usepackage{amsmath}
\usepackage{amssymb}
\usepackage{esvect}

\makeatletter
\newcommand{\imagesource}[1]{{\centering\hfill\break\hbox{\scriptsize Image Source:\thinspace{\tiny\itshape #1}}\par}}
\newcommand{\image}[3][\@nil]{%
        \def\tmp{#1}%
        \begin{center}
        \ifx\tmp\@nnil
            \includegraphics[max height = 0.55\textheight, max width = \textwidth]{images/#2}
        \else
            \includegraphics[max height = 0.50\textheight, max width = \textwidth]{images/#2}
            \linebreak
            #1
        \fi
        \linebreak
        {\tiny Image Source:\thinspace{\tiny #3}}
        \end{center}
}

\newenvironment{code}{%
 \VerbatimEnvironment
 \begin{adjustbox}{max width=\textwidth, max height=0.7\textheight}
 \begin{BVerbatim}
  }{
  \end{BVerbatim}
 \end{adjustbox}
}

\title{06 - Topic Selection and Literature Review}


\author{Robert Lowe}

\institute[Southeast Missouri State University] % (optional, but mostly needed)
{
  Department of Computer Science\\
  Southeast Missouri State University
}

\date[]{}
\subject{}

\pgfdeclareimage[height=1.0cm]{university-logo}{images/semo-logo}
\logo{\pgfuseimage{university-logo}}



\AtBeginSection[]
{
  \begin{frame}<beamer>{Outline}
    \tableofcontents[currentsection]
  \end{frame}
}


\begin{document}

\begin{frame}
  \titlepage
\end{frame}

\begin{frame}{Outline}
  \tableofcontents
\end{frame}


% Structuring a talk is a difficult task and the following structure
% may not be suitable. Here are some rules that apply for this
% solution: 

% - Exactly two or three sections (other than the summary).
% - At *most* three subsections per section.
% - Talk about 30s to 2min per frame. So there should be between about
%   15 and 30 frames, all told.

% - A conference audience is likely to know very little of what you
%   are going to talk about. So *simplify*!
% - In a 20min talk, getting the main ideas across is hard
%   enough. Leave out details, even if it means being less precise than
%   you think necessary.
% - If you omit details that are vital to the proof/implementation,
%   just say so once. Everybody will be happy with that.
\section{Topic Selection}
\begin{frame}{Finding Ideas}
    \begin{columns}
    \column{0.5\textwidth}
    \begin{itemize}
        \item Researchers are always reading!
        \item Read journals, magazines, books, articles, etc.
        \item Learn the current trends in research, and identify ideas.
        \item Could we implement this?
        \item Could we improve that?
    \end{itemize}
    \column{0.5\textwidth}
    \begin{center}
        \image{research.jpg}{\href{https://www.formpl.us/blog/research-report}{formpl.us}}
    \end{center}
    \end{columns}
\end{frame}

\begin{frame}{Example: Hierarchical Hopfield Networks}
    \begin{columns}
    \column{0.5\textwidth}
        \begin{itemize}
            \item Kurzweil proposes hierarchical pattern matching as the basis of intelligence.
            \item I think ``Could this scheme be implemented using hierarchical hopfield networks?''
            \item I then set out to write down the idea in my notebook.
        \end{itemize}
    \column{0.5\textwidth}
        \image{createAMind}{\href{https://www.amazon.com/How-Create-Mind-Thought-Revealed-ebook/dp/B007V65UUG/ref=sr_1_1?dchild=1&keywords=how+to+create+a+mind&qid=1625574689&sr=8-1}{Amazon}}
    \end{columns}
\end{frame}

\begin{frame}{Your Initial Idea}
    One or more of several things could be true about your initial idea:
    \begin{itemize}
        \item The idea could already exist in the literature.
        \item The idea could be a refinement of existing ideas.
        \item The idea could be entirely novel.
        \item The idea could be fruitless. (Failure is always an option.)
    \end{itemize}
\end{frame}

\section{Literature Review}
\begin{frame}{Literature Review Goals}
    \begin{enumerate}
        \item Identify how your idea fits with current trends in research.
        \item Find related articles.
        \item Find articles which help refine your approach.
        \item Find articles which both support and refute your approach.
        \item Determine to what extent your idea is new.
        \item Understand the current state of the art.
        \item Refine your idea so that it is:
        \begin{itemize}
            \item Implementable
            \item Testable
        \end{itemize}
        \item Identify what your specific contributions will be.
    \end{enumerate}
\end{frame}

\begin{frame}{Performing a Literature Review}
    \begin{enumerate}
        \item Find a few recent articles pertaining to your research.
        \item As you read through your articles:
        \begin{itemize}
            \item Annotate and gather notes.
            \item Identify new keywords to search.
            \item Identify past sources that you should likely read.
        \end{itemize}
        \item Continue searching new documents using:
        \begin{itemize}
            \item Your new keywords.
            \item Citations in the articles you are reading.
            \item Articles which cite the articles you are reading.
            \item Build a tree of articles, so you can be sure of fully reviewing your topic!
        \end{itemize}
    \end{enumerate}
\end{frame}

\begin{frame}{Literature Review Outcomes}
    \begin{itemize}
        \item Write the introduction to your paper.
        \item Write the related works section of your paper.
        \item Identify a set of goals for your research.
        \item Begin the process of creating a research proposal (if needed).
    \end{itemize}
\end{frame}
\end{document}

